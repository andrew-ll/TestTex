\documentclass[11pt]{article}

\setlength{\topmargin}{-.2in}
\setlength{\textheight}{8.6in}
\setlength{\oddsidemargin}{.125in}
\setlength{\textwidth}{6.25in}
\setlength{\parindent}{0in}

\usepackage{color}
\usepackage{bbm}
\usepackage{url}
\usepackage{soul}
\usepackage{natbib}
\usepackage{pdfpages}
\usepackage{fancyhdr}
\pagestyle{fancy}
\fancyhead{}
\fancyfoot{}
			
\lhead{IE535 -- Linear Programming, Fall 2014}
\rhead{Homework 5 -- Due \textcolor{red}{October 27 (Monday), 2014}}
\rfoot{\thepage}

\newcommand{\minimize}[1]{\displaystyle\minim_{#1}}
\newcommand{\minim}{\mathop{\hbox{\rm minimize}}}
\newcommand{\maximize}[1]{\displaystyle\maxim_{#1}}
\newcommand{\maxim}{\mathop{\hbox{\rm maximize}}}
\newcommand{\sbjt}{\mathrm{subject\ to}}
\newcommand{\Cal}{\mathcal}
\newcommand{\Real}{\mathbb R}
\newcommand{\where}{\mathrm{where}}
\newcommand{\Hif}{{H}_{i, f}}
\newcommand{\N}{\mathcal{N}}
\newcommand{\J}{\mathcal{J}}
\newcommand{\free}{\mathrm{free}}
\newcommand{\1}{\mathbf{1}}
\newcommand{\indicator}[1]{\mathbbm{1}_{\left[ {#1} \right] }}
\mathchardef\mhyphen="2D

\renewcommand{\arraystretch}{2.0}

\begin{document}

Textbook [BJS10] (4th edition): 
\begin{enumerate}
\item[-] [3.5] (Note the objective function is to maximize. Though you can directly apply the simplex method to this problem, to prevent errors, 
it is advised that you convert the objective function into a minimization problem first.)\\

\item[-] [3.7] (\textcolor{blue}{Note: For the question ``Show that this approach is valid in this problem", 
it means that you need to explain why evaluating the objective function at all extreme points would give you the optimal solution of this problem. })\\

\item[-] [3.11] \\[10pt]

\end{enumerate} 

\textbf{Extra credit problem.} [BT97] 3.7

\end{document}